\documentclass{beamer}
\usetheme{metropolis}
\usepackage{xcolor}

% Supabase color palette
\definecolor{supabasegreen}{HTML}{3ECF8E}
\definecolor{supabasedark}{HTML}{1F1F1F}

% Dark theme settings
\setbeamercolor{background canvas}{bg=supabasedark}
\setbeamercolor{normal text}{fg=white,bg=supabasedark}
\setbeamercolor{frametitle}{fg=white,bg=supabasegreen}
\setbeamercolor{progress bar}{fg=supabasegreen}
\setbeamercolor{title}{fg=supabasegreen}
\setbeamercolor{structure}{fg=supabasegreen}

\usepackage{graphicx}
\usepackage{booktabs}
\usepackage{hyperref}

\title{Menu Generation Assistant}
\subtitle{Flutter + Supabase App}
\author{Haotian Sun, Dominic Pöltl, Joshua Lympany \\ \textbf{University of Tübingen}} 
\date{23.05.2025}
\institute{Department of Marketing}

\begin{document}

\begin{frame}
    \titlepage
\end{frame}

\begin{frame}{Outline}
    \tableofcontents
\end{frame}

% ------------------ FRONT END ------------------
\section{Front End}

\begin{frame}{App Purpose}
    \begin{block}{\textbf{Image-to-Recipe Flutter App}}
        \begin{itemize}
            \item Turns a food image into a custom recipe using AI
            \item Fully automated pipeline: Upload → Extract → Generate → Display
            \item Built using Flutter with multi-screen navigation
            \item Emphasis on intuitive UX and fast processing
        \end{itemize}
    \end{block}
\end{frame}

\begin{frame}{User Flow}
    \begin{block}{\textbf{4-Stage Pipeline}}
        \begin{itemize}
            \item \textbf{Upload}: User selects a food image
            \item \textbf{Extract}: Ingredients extracted from image
            \item \textbf{Generate}: Recipe created using ingredients
            \item \textbf{Display}: Recipe shown in structured format
        \end{itemize}
    \end{block}
\end{frame}

\begin{frame}{Page Structure}
    \begin{block}{\textbf{Modular Screen Architecture}}
        \begin{itemize}
            \item Each step implemented as a separate Flutter page
            \item Navigation via named routes
            \item Clear state transfer across stages
            \item Encourages scalability and decoupled logic
        \end{itemize}
    \end{block}
\end{frame}

\begin{frame}{Potential Applications}
    \begin{itemize}
        \item AI-driven cooking assistant
        \item Educational nutrition tool
        \item Foundation for food tracking or meal planning
        \item Could extend to barcode or text input
    \end{itemize}
\end{frame}

% ------------------ BACK END ------------------
\section{Back End}

\begin{frame}{Back-End Overview}
    \begin{block}{\textbf{Cloud-Driven Intelligence}}
        \begin{itemize}
            \item Handles image storage, user data, and AI inference
            \item Combines Supabase (database/storage/auth) with Google AI (vision + language)
            \item All processing offloaded to cloud for scalability
        \end{itemize}
    \end{block}
\end{frame}

\begin{frame}{Supabase Services}
    \begin{itemize}
        \item \textbf{Storage}: Uploaded food images stored securely
        \item \textbf{Database}: Stores user history, ingredients, recipes
        \item \textbf{Authentication}: Optional login support for personalization
        \item Fast RESTful API interface via Supabase SDK
    \end{itemize}
\end{frame}

\begin{frame}{Google AI APIs}
    \begin{itemize}
        \item \textbf{Vision AI}: Extracts ingredient keywords from image
        \item \textbf{Gemini / PaLM API}: Generates recipe text from ingredients
        \item Zero-shot prompting to adapt to various food types
        \item Reliable, scalable inference with minimal backend setup
    \end{itemize}
\end{frame}

\begin{frame}{Security \& Scalability}
    \begin{itemize}
        \item Supabase Row-Level Security for user-specific data
        \item Image access is signed and time-limited
        \item Backend logic can scale with serverless functions
        \item Easy to add monitoring or logging via Supabase or Google Cloud
    \end{itemize}
\end{frame}

\end{document}
